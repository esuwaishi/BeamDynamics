\documentclass[book]{jlreq}

\usepackage{graphicx}
\usepackage{amsmath,amssymb,amsthm}
\usepackage{siunitx}
\usepackage{physics}
\usepackage{bm}

%\renewcommand{\today}{\the\year/\the\month/\the\day}
%\renewcommand{\contentsname}{Contents}
%\renewcommand{\refname}{References}
\renewcommand{\figurename}{Fig.~}
\renewcommand{\tablename}{Table~}

\begin{document}
\title{Beam Dynamics}
\author{Shin-ichi YOSHIMOTO}
\maketitle
\tableofcontents
\clearpage

\part{数学的準備}
\chapter{微分形式}
\section{外形式}
\subsection{1-形式 (線形汎関数)}
線型汎函数(linear functional)は、ベクトル空間からその係数体への線型写像をいう\cite{Shiga}。線型形式(linear form)若しくは一次形式(one-form)\cite{Fukaya}あるいは余ベクトル(covector)ともいう\cite{Arnold}。

線形(汎)関数$\omega:\mathbf{R^n} \rightarrow\mathbf{R},\quad
\forall\lambda_1,\;\lambda_2 \in \mathbf{R},\quad\bm{\xi}_1,\;\bm{\xi}_2 \in \mathbf{R}^n$
%
\begin{equation}
    \omega(\lambda_1\bm{\xi}_1+\lambda_2\bm{\xi}_2)= \lambda_1\omega(\bm{\xi}_1)+\lambda_2\omega(\bm{\xi}_2)
\end{equation}
%
\subsection{2-形式}
$\omega^2:\mathbf{R}^n\times\mathbf{R}^n \rightarrow\mathbf{R},\quad
\forall\lambda_1,\;\lambda_2 \in \mathbf{R},\quad\bm{\xi}_1,\;\bm{\xi}_2,\;\bm{\xi}_3 \in \mathbf{R}^n$
%
\begin{align}
    \omega^2(\lambda_1\bm{\xi}_1+\lambda_2\bm{\xi}_2,\bm{\xi_3})
    &= \lambda_1\omega^2(\bm{\xi}_1,\bm{\xi_3})+\lambda_2\omega^2(\bm{\xi}_2,\bm{\xi_3})\\
    \omega^2(\bm{\xi}_1,\bm{\xi_2})&=-\omega^2(\bm{\xi}_2,\bm{\xi_1})
\end{align}

\subsection{k-形式}
\subsubsection{p-形式}
\paragraph{外積}
\subparagraph{微分形式}\cite{Yamamoto}
\section{グリーンの公式}
\begin{equation}
    \int\int_D\left(\frac{\partial Q}{\partial x}-\frac{\partial P}{\partial y}\right)dx dy = \int_C P dx + Q dy
\end{equation}
%
% \begin{thebibliography}{9}
%     \bibitem{Arnold}
%     アーノルド, 古典力学の数学的方法, 岩波書店, 2003
% \end{thebibliography}
%
\bibliographystyle{jplain}
\bibliography{references}
%
\end{document}